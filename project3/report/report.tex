\documentclass[11pt]{article}
\usepackage{../../latex/preamble}

\begin{document}
  % make title page
\begin{titlepage}
  \newcommand{\HRule}{\rule{\linewidth}{0.5mm}}
  \center
  \textsc{\LARGE Universitetet i Oslo}\\[1.5cm] % Name of your university/college
  \textsc{\Large }\\[0.5cm] % Major heading such as course name
  \textsc{\large FYS3150}\\[0.5cm] % Minor heading such as course title
  \HRule \\[0.4cm]
  { \huge \bfseries En sammenligning av numeriske integrasjonsmetoder, Gauss kvadratur og MonteCarlo}\\[0.4cm] % Title of your document
  \HRule \\[1.5cm]
  \Large \emph{Skrevet av:}\\
  Lyder \textsc{Rumohr Blingsmo} og Bendik \textsc{Samseth}\\[3cm]
  {\large \today}\\[3cm]
  \vfill
\end{titlepage}

\begin{abstract}
  I denne rapporten har vi beregnet forventingsverdien til korrelasjonsenergien
  mellom to elektroner i bane rundt en heliumkjerne. Beregningen
  innebærer å løse et 6-dimensjonalt integral. Vi bruker
  Gauss-Legendre- og Gauss-Laguerre-kvadratur, samt
  Monte Carlo-integrasjon med og uten ``importance sampling''. Vi
  konkluderer med at Gauss-Legendre og standard Monte Carlo brukt
  blindt på problemstillingen fungerer mindre bra, men med litt
  tenkearbeid får vi gode resultater med Gauss-Laguerre og Monte Carlo
  med ``importance sampling''.  
\end{abstract}

\section{Innledning}

Vi antar at bølgefunksjonen til hvert elektron kan skrives som en ett-partikkel
bølgefunksjon for et elektron i et hydrogenatom. Single-particle bølgefunksjonen for 
grunntilstanden er da gitt ved en dimensjonsløs variabel
\[
   {\bf r}_i =  x_i {\bf e}_x + y_i {\bf e}_y +z_i {\bf e}_z ,
\]
as
\[
   \psi_{1s}({\bf r}_i)  =   e^{-\alpha r_i},
\]
Her er ikke $\psi_{1s}$ normalisert. $\alpha$ er en parameter og
\[
r_i = \sqrt{x_i^2+y_i^2+z_i^2}.
\]
Vi setter $\alpha=2$, som tilsvarer ladningen til et heliumatom $Z=2$. 

Antar videre at bølgefunksjonen for to elektroner da kan skrives som et produkt av
to $1s$ bølgefunksjoner:
\[
   \Psi({\bf r}_1,{\bf r}_2)  =   e^{-\alpha (r_1+r_2)}.
\]

Integralet vi skal løse er den kvantemekaniske forventningsverdien
til korrelasjonsenergien til to elektroner som frastøter hverandre via et Coulumb-potensial
\begin{equation}\label{eq:correlationenergy}
   \langle \frac{1}{|{\bf r}_1-{\bf r}_2|} \rangle =
   \int_{-\infty}^{\infty} d{\bf r}_1d{\bf r}_2  e^{-2\alpha (r_1+r_2)}\frac{1}{|{\bf r}_1-{\bf r}_2|}.
\end{equation}

Her vet vi at integralet \eqref{eq:correlationenergy} har eksakt løsning $5\pi^2/16^2$. Vi skal tilnærme denne 
løsningen på fire forskjellige måter og sammenligne resultatene både i presisjon og kjøretid.

\section{Metoder}
Vi anvender to forskjellige metoder for Gauss-kvadratur og to forskjellige MC-metoder.
\subsection{Gauss-kvadratur}
\subsubsection{Rett fram med Legendre-polynomer}
Vi skriver om \eqref{eq:correlationenergy} på formen

\begin{align}\label{eq:xyzINT} 
  \langle \frac{1}{|{\bf r}_1-{\bf r}_2|} \rangle =
   \int_{-\infty}^{\infty} \dots \int_{-\infty}^{\infty}
   \frac{d{x}_1d{x}_2d{y}_1d{y}_2d{z}_1d{z}_2  
   e^{-2\alpha (x_1^2+x_2^2+y_1^2+y_2^2+z_1^2+z_2^2)}}{ \sqrt{{(x_1-x_2)}^2+{(y_1-y_2)}^2+{(y_1-y_2)}^2}}
\end{align}

Her ligger $x1$, $x2$, $y1$, $y2$, $z1$ og $z2$ på intervallet 
$[-\infty, \infty]$. Dette er åpenbart umulig å representere numerisk
og vi må derfor velge et intervall $[a,b]$ som vi mapper med Legendre-polynomer.
Mer om disse polynomene \cite{Lecture-notes}. I vårt tilfelle
har vi valgt intervallet $[-2,2]$ etter vurderinger av figur \ref{fig:PSI}.

\begin{figure}[ht]
  \centering
  \includegraphics[scale=0.7]{../fig/psiPlot.png}
  \caption{\label{fig:PSI} Et plot av  $e^{-2\alpha (r_1+r_2)}$. Vi må velge
et intervall $[a,b]$ slik at vi får med hovedtrekkene. Vi har valgt 
intervallet $[-2,2]$. Siden vi ikke kan velge antall punkter, N,
alt for stort, må vi også ta hensyn til dette i valget av intervall.}
\end{figure}


\subsubsection{Koordinatskift og Laguerre-polynomer}
Her begynner vi med å gjøre et koordinatskifte av 
\eqref{eq:correlationenergy} til sfæriske koordinater. Vi får da variablene

\[
   d{\bf r}_1d{\bf r}_2  = r_1^2dr_1 r_2^2dr_2 dcos(\theta_1)dcos(\theta_2)d\phi_1d\phi_2,
\]
og
\[
   \frac{1}{r_{12}}= \frac{1}{\sqrt{r_1^2+r_2^2-2r_1r_2cos(\beta)}}
\]
og 
\[
cos(\beta) = cos(\theta_1)cos(\theta_2)+sin(\theta_1)sin(\theta_2)cos(\phi_1-\phi_2))
\]

Det resulterende integralet blir

\begin{align}
\int_{0}^{\infty} r_1^2 dr_1 \int_{0}^{\infty} r_2^2 dr_2
\int_{0}^{\pi} dcos(\theta_1) \int_{0}^{\pi} dcos(\theta_2)
\int_{0}^{2\pi} d\phi_1 \int_{0}^{2\pi} d\phi_2 \frac{e^{-2\alpha (r_1+r_2)}}{r_{12}}
\end{align}

med

\[
dcos(\theta_1) = -sin(\theta_1)d\theta
\]

Her mapper vi $\phi_1$, $\phi_2$ $\theta_1$ og $\theta_2$ med Legendre-polynomer slik som i sted.
Men $ r \in [0, \infty] $ med Laguerre-polynomer \cite{Lecture-notes}.


\subsection{Monte Carlo}
\subsubsection{Brute force Monte Carlo}
Nok en gang bruker vi \eqref{eq:xyzINT} som utgangspunkt. Vi velger seks koordinater( $x1$, $x2$, $y1$, $y2$, $z1$ og $z2$) uniformt fordelt og tilfeldig i intervallet $[a,b]$. Dette gjør vi $N$ ganger.
For hver gang $n_i$ regner vi ut integranden og summerer disse. Til slutt deler vi på antall punkter
, d.v.s ganger med bredden på intervallet. Vi håper på at vi ved å velge tilfeldige punkter
skal klare å fange opp oppførselen til funksjonen tilstrekkelig godt.
\subsubsection{Importance sampling Monte Carlo}

\section{Resultater}

\section{Konklusjon}



\printbibliography
\end{document}
%%% Local Variables:
%%% mode: latex
%%% TeX-master: t
%%% End:
