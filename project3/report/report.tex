\documentclass[11pt]{article}
\usepackage{../latex/preamble}

\begin{document}
  % make title page
\begin{titlepage}
  \newcommand{\HRule}{\rule{\linewidth}{0.5mm}}
  \center
  \textsc{\LARGE Universitetet i Oslo}\\[1.5cm] % Name of your university/college
  \textsc{\Large }\\[0.5cm] % Major heading such as course name
  \textsc{\large FYS3150}\\[0.5cm] % Minor heading such as course title
  \HRule \\[0.4cm]
  { \huge \bfseries En sammenligning av numeriske integrasjonsmetoder, Gauss kvadratur og MonteCarlo}\\[0.4cm] % Title of your document
  \HRule \\[1.5cm]
  \Large \emph{Skrevet av:}\\
  Lyder \textsc{Rumohr Blingsmo} og Bendik \textsc{Samseth}\\[3cm]
  {\large \today}\\[3cm]
  \vfill
\end{titlepage}

\begin{abstract}
  I denne rapporten har vi beregnet forventingsverdien til korrelasjonsenergien
  mellom to elektroner i bane rundt en heliumkjerne. Beregningen
  innebærer å løse et 6-dimensjonalt integral. Vi bruker
  Gauss-Legendre- og Gauss-Laguerre-kvadratur, samt
  Monte Carlo-integrasjon med og uten ``importance sampling''. Vi
  konkluderer med at Gauss-Legendre og standard Monte Carlo brukt
  blindt på problemstillingen fungerer mindre bra, men med litt
  tenkearbeid får vi gode resultater med Gauss-Laguerre og Monte Carlo
  med ``importance sampling''.  
\end{abstract}

\section{Innledning}

Vi antar at bølgefunksjonen til hvert elektron kan skrives som en ett-partikkel
bølgefunksjon for et elektron i et hydrogenatom. Single-particle bølgefunksjonen for 
grunntilstanden er da gitt ved en dimensjonsløs variabel
\[
   {\bf r}_i =  x_i {\bf e}_x + y_i {\bf e}_y +z_i {\bf e}_z ,
\]
as
\[
   \psi_{1s}({\bf r}_i)  =   e^{-\alpha r_i},
\]
Her er ikke $\psi_{1s}$ normalisert. $\alpha$ er en parameter og
\[
r_i = \sqrt{x_i^2+y_i^2+z_i^2}.
\]
Vi setter $\alpha=2$, som tilsvarer ladningen til et heliumatom$Z=2$. 

Antar videre at bølgefunksjonen for to elektroner da kan skrives som et produkt av
to $1s$ bølgefunksjoner:
\[
   \Psi({\bf r}_1,{\bf r}_2)  =   e^{-\alpha (r_1+r_2)}.
\]

Integralet vi skal løse er den kvantemekaniske forventningsverdien
til korrelasjonsenergien til to elektroner som frastøter hverandre via et Coulumb-potensial
\begin{equation}\label{eq:correlationenergy}
   \langle \frac{1}{|{\bf r}_1-{\bf r}_2|} \rangle =
   \int_{-\infty}^{\infty} d{\bf r}_1d{\bf r}_2  e^{-2\alpha (r_1+r_2)}\frac{1}{|{\bf r}_1-{\bf r}_2|}.
\end{equation}

Her vet vi at integralet \eqref{eq:correlationenergy} har eksakt løsning $5\pi^2/16^2$. Vi skal tilnærme denne 
løsningen på fire forskjellige måter og sammenligne resultatene både i presisjon og kjøretid.



\printbibliography
\end{document}
%%% Local Variables:
%%% mode: latex
%%% TeX-master: t
%%% End:
