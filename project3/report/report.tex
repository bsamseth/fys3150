\documentclass[11pt]{article}
\usepackage{../latex/preamble}

\begin{document}
  % make title page
\begin{titlepage}
  \newcommand{\HRule}{\rule{\linewidth}{0.5mm}}
  \center
  \textsc{\LARGE Universitetet i Oslo}\\[1.5cm] % Name of your university/college
  \textsc{\Large }\\[0.5cm] % Major heading such as course name
  \textsc{\large FYS3150}\\[0.5cm] % Minor heading such as course title
  \HRule \\[0.4cm]
  { \huge \bfseries En titt på Jacobis metode og noen anvendelser på enkle Schr\"odingerligninger}\\[0.4cm] % Title of your document
  \HRule \\[1.5cm]
  \Large \emph{Skrevet av:}\\
  Lyder \textsc{Rumohr Blingsmo} og Bendik \textsc{Samseth}\\[3cm]
  {\large \today}\\[3cm]
  \vfill
\end{titlepage}

\begin{abstract}
  I denne rapporten har vi beregnet forventingsverdien til korrelasjonsenergien
  mellom to elektroner i bane rundt en heliumkjerne. Beregningen
  innebærer å løse et 6-dimensjonalt integral. Vi bruker
  Gauss-Legendre- og Gauss-Laguerre-kvadratur, samt
  Monte Carlo-integrasjon med og uten ``importance sampling''. Vi
  konkluderer med at Gauss-Legendre og standard Monte Carlo brukt
  blindt på problemstillingen fungerer mindre bra, men med litt
  tenkearbeid får vi gode resultater med Gauss-Laguerre og Monte Carlo
  med ``importance sampling''.  
\end{abstract}

Sitat \cite{Lecture-notes}.

\printbibliography
\end{document}
%%% Local Variables:
%%% mode: latex
%%% TeX-master: t
%%% End:
