\documentclass[11pt]{article}
\usepackage{../../latex/preamble}

\begin{document}
  % make title page
\begin{titlepage}
  \newcommand{\HRule}{\rule{\linewidth}{0.5mm}}
  \center
  \textsc{\LARGE Universitetet i Oslo}\\[1.5cm] % Name of your university/college
  \textsc{\Large }\\[0.5cm] % Major heading such as course name
  \textsc{\large FYS3150}\\[0.5cm] % Minor heading such as course title
  \HRule \\[0.4cm]
  { \huge \bfseries Ising-model blabla}\\[0.4cm]
  \HRule \\[1.5cm]
  \Large \emph{Skrevet av:}\\
  Lyder \textsc{Rumohr Blingsmo} og Bendik \textsc{Samseth}\\[3cm]
  {\large \today}\\[3cm]
  \vfill
\end{titlepage}

\begin{abstract}
I denne rapporten skal vi se på Ising-modellen i to dimensjoner. Det vil si
et rutenett av $n \times n $ partikler, der alle partiklene enten har
spinn opp, $\uparrow$ eller spinn ned, $\downarrow$. Spesielt ser vi
på de termodynamiske egenskapene til et slikt system. Vi bruker
Metropolis-algoritmen med 'periodic boundary conditions'. Alt materiale 
som refereres er tilgjengelig på~\cite{github-repo}. 
\end{abstract}

\section{Innledning}
\label{sec:innledning}
Ising-modellen er et system av $n \times n$ partikler ordnet i et rutenett. 
Hver partikkel kan være i én av to tilstander, enten spinn opp, $\uparrow$,
eller spinn ned, $\downarrow$. Energien blir da: 

\begin{equation}
  E=-J\sum_{<kl>}^{N}s_ks_l
\end{equation}
med  $s_k=\pm 1$, $N$ er antall partikler.
$J$ er en konstant som beskriver styrken på interaksjonen mellom
nabopartikler. Symbolet $<kl>$ betyr at vi summerer over bare de
nærmeste naboene. Vi antar også  ferromagnetisme, dvs.  $J> 0$. Vi bruker også
periodiske randbetingelser. Det betyr at partiklene langs randa 'ser' 
partiklene langs motsatte rand.

For å få en intuisjon om hvordan et slikt system fungerer ser vi først på 
$2 \times 2$-tilfellet. 



\section{Resultater}
\section{Konklusjon}

\clearpage
\printbibliography
\end{document}
%%% Local Variables:
%%% mode: latex
%%% TeX-master: t
%%% End:
