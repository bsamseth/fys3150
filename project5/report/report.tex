\documentclass[11pt, twocolumn]{article}
\usepackage{../../latex/preamble}

\newcommand{\abs}[1]{|#1|}
\begin{document}
  % make title page
\begin{titlepage}
  \newcommand{\HRule}{\rule{\linewidth}{0.5mm}}
  \center
  \textsc{\LARGE Universitetet i Oslo}\\[1.5cm] % Name of your university/college
  \textsc{\Large }\\[0.5cm] % Major heading such as course name
  \textsc{\large FYS3150}\\[0.5cm] % Minor heading such as course title
  \HRule \\[0.4cm]
  { \huge \bfseries N-body simulering}\\[0.4cm]
  \HRule \\[1.5cm]
  \Large \emph{Skrevet av:}\\
  Lyder \textsc{Rumohr Blingsmo} (k.nr. 2) og Bendik \textsc{Samseth} (k.nr. 27)\\[3cm]
  {\large \today}\\[3cm]
  \vfill
\end{titlepage}
\twocolumn[
\begin{@twocolumnfalse}
\tableofcontents
\vspace{\baselineskip}
\begin{abstract}
I denne rapporten utvikler vi en N-body modell. Det vil si et system av
$N$ masser som vekselvirker kun ved gravitasjon. Spesielt studerer vi kollaps
av et slik system når alle partiklene begynner i ro. Vi sammenligner 
stabiliteten til to forskjellige løsningsmetoder, RungeKutta4 og VelocityVerlet,
ved å kikke på energibevaring i systemet. 
Alt materiale som har blitt referert er tilgjengelig
 på Github~\cite{github-repo}. 

\end{abstract}
\vspace{\baselineskip}
\end{@twocolumnfalse}
]

\section{Innledning}
Vi skal se på et stjernekluster med $N$ partikler. I vår modell tenker
vi oss at hver partikkel representerer en eller noen få stjerner. Vi
skal også anta at gravitasjon er den eneste kraften som virker:

\begin{align}
 \vec F = -\frac{GM_1M_2}{r^3}\vec r.\
\end{align}

Her er $G$ gravitasjonskonstanten, $M_i$ er massene og $r$ er avstanden mellom dem. 
Denne antagelsen er god under forutsetningen av at $r$ er relativt
stor slik at  ingen kontaktkrefter gjør seg gjeldene. Vi bruker
Newtons andre lov, og får differensialligningene 

\begin{align} \label{eq:a_i}
a_i = \sum_{j \neq i} \frac{GM_j}{{|\vec{r_i} - \vec{r_j}|}^2} , \ \ \
  i = 1,2, \cdots N 
\end{align}

der $a_i$ er akselerasjonen til masse $i$. Dette er et system av
ordinære differensialligninger, noe vi kan løse ved å anvende
numeriske metoder. I denne rapporten ser vi på RungeKutta4 og Velocity-Verlet.

\subsection{Kort om RungeKutta4~\small{\cite{RK4}}}
Metoden antar en funksjon på formen 
\begin{align*}
\dot{y} = f(t, y), \quad y(t_0) = y_0,
\end{align*}

som i vårt tilfelle er gitt ved \eqref{eq:a_i} med 

\begin{align*}
\dot y = \dot{v_i} = a_i 
\end{align*}
Vi trenger initialbetingelser for hastighet,
\begin{align*}
 v_i(t_0) = v_{i,0}.
\end{align*}
Vi velger en steglengde $h>0$ og definer 

\begin{align*}
y_{n+1} &= y_n + \tfrac{h}{6}\left(k_1 + 2k_2 + 2k_3 + k_4 \right)\\
t_{n+1} &= t_n + h \\
\end{align*}
der vi har
\begin{align*}
k_1 &= f(t_n, y_n), \\
k_2 &= f(t_n + \tfrac{h}{2}, y_n + \tfrac{h}{2} k_1), \\
k_3 &= f(t_n + \tfrac{h}{2}, y_n + \tfrac{h}{2} k_2), \\
k_4 &= f(t_n + h, y_n + hk_3).
\end{align*}

Vi gjentar så metoden en gang til for 

\begin{align*}
\dot{x_i} = v_i.
\end{align*}
I realiteten er det mulig å vektorisere prosessen slik at vi løser
både hastigheten og posisjonen til alle partiklene samtidig.

\subsection{Kort om Velocity-Verlet~\small{\cite{Velocity-Verlet}}}
\label{sec:om-VV}
Vi anvender også Velocity-Verlet på \eqref{eq:a_i}. Denne metoden fungerer for tilfeller der 
akselerasjonen ikke er en funksjon av hastigheten. 
Algoritmen for metoden er gitt som følger:

\begin{enumerate}
\item Regn ut: \begin{align*}
\vec{v}\left(t + \tfrac12\,\Delta t\right) = \vec{v}(t) + \tfrac12\,\vec{a}(t)\,\Delta t\ 
\end{align*}


\item Finn så: \begin{align*} \vec{r}(t + \Delta t) = \vec{r}(t) + \vec{v}\left(t + \tfrac12\,\Delta t\right)\, \Delta t \end{align*}


\item Regn ut: \begin{align*} \vec{a}(t + \Delta t) 
\end{align*} fra \eqref{eq:a_i} med posisjonen $ \vec{r}(t + \Delta t) $


\item Til slutt: \begin{align*} \vec{v}(t + \Delta t) = \vec{v}\left(t + \tfrac12\,\Delta t\right) + \tfrac12\,\vec{a}(t + \Delta t)\Delta t \end{align*}
\end{enumerate}

\subsection{Initialisering av systemet}
Vi kan selv velge fritt starttilstanden til systemet vårt. Vi velger å
begynne med $N$ objekter fordelt uniformt innenfor en sfære med
radius $R_0$. Objektene gir vi en tilfeldig masse ved å la massene
følge en Gaussisk fordeling med snittverdi $10M_\odot$ og standardavik
på $M_\odot$. Her er $M_\odot$ solmassen. Vi lar alle starte i ro. 

Av disse startbetingelesene er det en del som trenger noe ekstra
oppmerksomhet\footnote{Det som følger er i stor grad en oversettelse
  av et skriv laget av H. T. Ihle som ble sendt ut via epost (ikke
  listet som en referanse fordi den ikke kan vises til noe sted).}. Det er ikke helt fullstendig trivielt å lage uniformt
fordelte koordinater innenfor en sfære. Problemet er at hvis vi lar
objektene være uniformt fordelt i $r$, så vil vi ende med en mye
større tetthet rundt midten av sfæren. Og hvis vi fordeler objektene
uniformt i $\theta$ vil tettheten blir mye større rundt polene. 

Vi løser dette ved å bruke at volumelementet skal være det samme i
alle koordinatsystemer. Vi innfører koordinatene $u, v, w \in [0,1]$
og kobler disse koordinatene mot (i rekkefølge) $r,\theta$ og
$\phi$. Vi kan da skrive 
\begin{align*}
  r^2\sin\theta dr d\theta d\phi = (abc)dudvdw
\end{align*}
der $A=abc$ er en konstant. Separerer vi likningen for hvert par av
koordinater får vi likningene 
\begin{align*}
  r^2dr = adu,\ \ \sin\theta d\theta = bdv,\ \ d\phi = cdw
\end{align*}
Begynner vi med den siste og enkleste får vi 
\begin{align}
  d\phi = cdw\Rightarrow \phi = 2\pi w
\end{align}
der vi har brukt at $\phi=2\pi$ må svare til $w=1$. Samme
fremgangsmåte for $r$ og $\theta$ gir
\begin{align}
  r &= R_0 \sqrt[3]{u}\\
  \theta &= \arccos(1-2v).
\end{align}
Vi får de kartesiske koordinatene ved vanlig overgang fra sfæriske
koordinater:
\begin{align*}
  x = r\sin\theta\sin\phi\\
y = r\sin\theta\sin\phi\\
z = r\cos\theta.
\end{align*}


\subsubsection{Forventninger}
Disse initialbetingelsene blir samlet kalt sfærisk
kald-kollaps~\cite{cold-collapse}. I grensen der $N\rightarrow\infty$,
med en konstant massetetthet $\rho_0$, vil vi se at legemene kollapser
sammen i en singulæritet og danner et kontinuerlig fluid. Det kan
vises at dete vil skje etter en tid~\cite{project5-exercise}
\begin{align}
  \tau_\text{crunch} = \sqrt{\frac{ 3\pi }{ 32G\rho_0 }}\label{eq:t-crunch}.
\end{align}
Dette blir dermed en veldig naturlig tidsenhet, og vi måler alle tider
i antall $\tau_\text{crunch}$. 

Vi vil derimot ikke forvente å se en singulæritet i vår
simulering. Dette er fordi vi ser på punktmasser som kun vekselvirker
gjennom gravitasjon, i tillegg til at vi selvsagt må begrense oss til
en endelig verdi for $N$. Hva vi derimot vil forvente å se er at noen
partikler blir slynget ut av systemet (unnslipper), mens resten blir
værende i en eller annen form for bane rundt et felles massesenter. 

\section{Implementering og testing}
Hovedprogrammet er skrevet i C++, mens vi har gjort dataanalysen i Python.
Alle programmer er tilgjengelig på Github~\cite{github-repo}. 

\subsection{Klassesystem}
Vi har to klasser. Én objektklasse 'Body' som inneholder med posisjon, fart 
og masse for et gitt legeme. Den andre klassen er 'Universe' og initialiseres
med et gitt antall 'Body'-objekter. Denne klassen inneholder mange metoder,
som f.eks. \texttt{Universe::energy} for å regne ut energien til systemet, og 
\texttt{Universe::solve\_RK4} eller \texttt{Universe::solve\_Verlet} for å løse systemet
for en gitt tid.

\subsection{Solver}
Selve implementasjonen av løsningsmetodene er noe innviklet. Dette
kommer av at vi har et system av to ordinære differensialligninger,
hver med tre koordinater. Altså har vi egentlig seks diff.likninger
som skal løses samtidig. Metodene som vi har skissert tidligere endrer
ikke form, men det kan bli litt vanskelig å holde styr på alt
sammen. Derfor skal vi gå nokså grundig i gjennom hvordan disse
metodene er implementert. Dvs., vi gjør dette for Runge-Kutta;
Velocity-Verlet er løst på en analog måte. RK4 er implementert i
\texttt{Universe} slik som vist i vedlegg \ref{lst:solver-RK4}. 

Vi begynner med å definere flere $N\times 7$ matriser. Dette svarer til
en rad per legeme, der kolonnene er de tilhørende verdiene av (i rekkefølge) $x,y,z,v_x,v_y,v_z$
og $M$. For eksempel matrisen $y_i$ er på formen
\begin{align*}
  y_i = \begin{pmatrix}
    x_1&y_1&z_1&v_{x_1}&v_{y_1} &v_{z_1} & m_1\\
    x_2&y_2&z_2&v_{x_2}&v_{y_2} &v_{z_2} & m_2\\
    \vdots&&\ddots&&&&\vdots\\
    x_N&y_N&z_N&v_{x_N}&v_{y_N} &v_{z_N} & m_N\\
  \end{pmatrix}
\end{align*}

Til å begynne med fyller vi $y_i$ med data fra den initielle tilstanden til systemet.
Vi går så inn i tidsløkken. Her brukes en funksjon
\texttt{derivative(y\_i, a\_i, n\_bodies)} som gjør at matrisen $a_i$ fylles med de
deriverte verdiene av matrise $y_i$. Det vil si at posisjonen i $y_i$
får tilsvarende element i $a_i$ satt som hastigheten, og hastigheten i
$y_i$ blir tilsvarende satt til akselerasjonen i $a_i$. 

\begin{align*}
  a_i = \begin{pmatrix}
    v_{x_1}&v_{y_1} &v_{z_1}&a_{x_1}&a_{y_1} &a_{z_1} & m_1\\
    v_{x_2}&v_{y_2} &v_{z_2}&a_{x_2}&a_{y_2} &a_{z_2} & m_2\\
    \vdots&&\ddots&&&&\vdots\\
    v_{x_N}&v_{y_N} &v_{z_N}&a_{x_N}&a_{y_N} &a_{z_N} & m_N\\
  \end{pmatrix}
\end{align*}

Med denne fremgangsmåten følges så algoritmen for Velocity-Verlet som
er skissert i seksjon \ref{sec:om-VV}. Samme type tanke blir brukt for
å implementere Rungekutta4.



\subsection{Første test av programmet}
Det er ikke alltid lett å sjekke om en løsning gir mening. Vi begynner med
å animere systemet \footnote{Tilgjengelig på \cite{github-repo} under navnet
animate\_pos.py}, og ser fort at det skjedde noe rart når massene kom veldig
nærme hverandre. Som en korreksjon til dette la vi til et feilledd i kraften

\begin{align}
F_{mod} = -\frac{GM_1M_2}{r^2 + \epsilon}.
\end{align}

Der $\epsilon$ gjør at $F_{mod}$ ikke går i lufta når $r^2 \rightarrow \infty$. 
Dette kan vi tenke på som at legemene har en endelig størrelse. 
Vi setter $\epsilon = 10^{-3}$. 

\subsection{Energibevaring}
Selv om animasjoner alltid er flott, tenker vi det kan være lurt å ha en mer
'matematisk' test på systemet vårt. Dette gjør oss også mer egnet til
å gjøre en kvalitativ vurdering av hvilken av de løsningnsmetodene som er best
I et slikt ideelt system med bare 
gravitasjon(konservativ kraft) er det naturlig å se på energibevaring. Vi får
energien

\begin{align}
E_{p} = - \sum_{i = 1}^{N} \sum_{j = i+1}^{N} \frac{GM_jM_i}{{||\vec{r_i} - \vec{r_j}||}} \\
E_{k} = \sum_{i} \frac{1}{2}M_i|\vec{v_i}|^2 \\
E_{tot} = E_{p} + E_{k}
\end{align}

I tabell \ref{tab:RKvsVV}


\begin{table*}[!ht]
\centering
\caption{En tabell som sammenligner relativ energiendring for de to metodene
RungeKutta4 og VelocityVerlet. Energiendringen regnes ut fra systemet starter
i ro ved $t_0 = 0$ til $T$ for to forskjellige steglengder
$h$.  Kjøretiden, $T$, måles i enhet $\tau_{crunch}$. Ideelt sett vil vi ha $\Delta E = 0$.}
\label{tab:RKvsVV}
\vspace{0.5cm}
\begin{tabular}{ccccc}
Kjøreparametre $(T, log(h))$ & $(0.79,-3)$ & $(0.79,-4)$ & $(2.5,-3)$ & $(2.5,-4)$ \\
\hline 
Energiendring og tidsforbruk,$(E,\Delta t)$ RK4 & $(2.4E-01, 006.6)$ & $(2.8E-03, 046.7)$ & $(6.6E-01, 011.1)$ & $(2.2E-02, 120.8)$ \\
Energiendring og tidsforbruk,$(E,\Delta t)$ VV & $(4.9E-01, 001.9)$ & $(2.9E-03, 027.5)$ & $(9.1E-01, 005.7)$ & $(5.1E-03, 058.6)$ \\
\hline
\end{tabular}
\end{table*}




\printbibliography
\onecolumn
\section{Vedlegg}
\begin{itemize}
  \item[]\lstinputlisting[language=c++, caption={C++ metode som bruker Runge-Kutta 4 til å tidsutvikle
  systemet.}, label={lst:solver-RK4}]{solversnippet.txt}
\end{itemize}

\end{document}
%%% Local Variables:
%%% mode: latex
%%% TeX-master: t
%%% End:
