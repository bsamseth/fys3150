\documentclass[11pt, twocolumn]{article}
\usepackage{../../latex/preamble}

\newcommand{\abs}[1]{|#1|}
\begin{document}
  % make title page
\begin{titlepage}
  \newcommand{\HRule}{\rule{\linewidth}{0.5mm}}
  \center
  \textsc{\LARGE Universitetet i Oslo}\\[1.5cm] % Name of your university/college
  \textsc{\Large }\\[0.5cm] % Major heading such as course name
  \textsc{\large FYS3150}\\[0.5cm] % Minor heading such as course title
  \HRule \\[0.4cm]
  { \huge \bfseries N-body simulering}\\[0.4cm]
  \HRule \\[1.5cm]
  \Large \emph{Skrevet av:}\\
  Lyder \textsc{Rumohr Blingsmo} (k.nr. 2) og Bendik \textsc{Samseth} (k.nr. 27)\\[3cm]
  {\large \today}\\[3cm]
  \vfill
\end{titlepage}
\twocolumn[
\begin{@twocolumnfalse}
\tableofcontents
\vspace{\baselineskip}
\begin{abstract}
I denne rapporten utvikler vi en N-body modell. Det vil si et system av
$N$ masser som vekselvirker kun ved gravitasjon. Spesielt studerer vi kollaps
av et slik system når alle partiklene begynner i ro. Vi sammenligner 
stabiliteten til to forskjellige løsningsmetoder, RungeKutta4 og VelocityVerlet,
ved å kikke på energibevaring i systemet. 
Alt materiale som har blitt referert er tilgjengelig
 på Github~\cite{github-repo}. 

\end{abstract}
\vspace{\baselineskip}
\end{@twocolumnfalse}
]

\section{Innledning}
Vi ser på en åpen galaksehop med $N$ partikler. I vårt system antar vi at 
gravitasjon er den eneste kraften som virker

\begin{align}
 F = -\frac{GM_1M_2}{r^2}.\
\end{align}

Der $G$ er gravitasjonskonstanten, $M$ er massene og $r$ er avstanden mellom dem. 
Denne antagelsen er god under forutsetningen av $r$ er relativt stor. Vi bruker
Newtons andre lov, og får differensialligningene 

\begin{align} \label{eq:a_i}
a_i = \sum_{j \neq i} \frac{GM_j}{{|\vec{r_i} - \vec{r_j}|}^2} ,  i = 1,2, \cdots N 
\end{align}

Der $a_i$ er akselerasjonen til masse $i$. Dette er en ordinære differensialligning, og
vi anvender RungeKutta4 og Velocity-Verlet.

\subsection{Kort om RungeKutta4 }
Gitt en funksjon på formen \cite{RK4}
\begin{align*}
\dot{y} = f(t, y), \quad y(t_0) = y_0
\end{align*}

I vårt tilfelle \eqref{eq:a_i} med 

\begin{align*}
\dot{v_i} = a_i  = \sum_{j \neq i} \frac{GM_j}{{|\vec{r_i} - \vec{r_j}|}^2} , i = 1,2, \cdots N \\
\end{align*}
med initialbetingelsene
\begin{align*}
r_i(t_0) = r_{i0}, v_i(t_0) = 0
\end{align*}
Velg en steglengde $h>0$ og definer 

\begin{align*}
y_{n+1} &= y_n + \tfrac{h}{6}\left(k_1 + 2k_2 + 2k_3 + k_4 \right)\\
t_{n+1} &= t_n + h \\
\end{align*}
der vi har
\begin{align*}
k_1 &= f(t_n, y_n), \\
k_2 &= f(t_n + \tfrac{h}{2}, y_n + \tfrac{h}{2} k_1), \\
k_3 &= f(t_n + \tfrac{h}{2}, y_n + \tfrac{h}{2} k_2), \\
k_4 &= f(t_n + h, y_n + hk_3).
\end{align*}

Vi gjentar så metoden en gang til for 

\begin{align*}
\dot{x_i} = v_i
\end{align*}

\subsection{Kort Velocity-Verlet}
Vi anvender også Velocity-Verlet på \eqref{eq:a_i}. Den fungerer for tilfeller der 
akselerasjonen ikke er en funksjon av hastigheten. 
Metoden er gitt på \cite{Velocity-Verlet}.

\begin{enumerate}
\item Calculate: \begin{align*}
\vec{v}\left(t + \tfrac12\,\Delta t\right) = \vec{v}(t) + \tfrac12\,\vec{a}(t)\,\Delta t\ 
\end{align*}


\item Calculate: \begin{align*} \vec{x}(t + \Delta t) = \vec{x}(t) + \vec{v}\left(t + \tfrac12\,\Delta t\right)\, \Delta t \end{align*}


\item Derive \begin{align*} \vec{a}(t + \Delta t) 
\end{align*} from the interaction potential using \begin{align*} \vec{x}(t + \Delta t) \end{align*}


\item Calculate: \begin{align*} \vec{v}(t + \Delta t) = \vec{v}\left(t + \tfrac12\,\Delta t\right) + \tfrac12\,\vec{a}(t + \Delta t)\Delta t \end{align*}
\end{enumerate}

\subsection{Initialisering av systemet}
Vi kan selv velge fritt starttilstanden til systemet vårt.


\section{Implementering og testing}
\subsection{Klassesystem}
\subsection{solver}




\printbibliography
\end{document}
%%% Local Variables:
%%% mode: latex
%%% TeX-master: t
%%% End:
