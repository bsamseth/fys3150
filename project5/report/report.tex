\documentclass[11pt]{article}
\usepackage{../../latex/preamble}

\usepackage{multicol}
\usepackage{media9}
\newtheorem*{viralthm}{Viralteoremet}
\newcommand{\abs}[1]{|#1|}
\begin{document}
  % make title page
\begin{titlepage}
  \newcommand{\HRule}{\rule{\linewidth}{0.5mm}}
  \center
  \textsc{\LARGE Universitetet i Oslo}\\[1.5cm] % Name of your university/college
  \textsc{\Large }\\[0.5cm] % Major heading such as course name
  \textsc{\large FYS3150}\\[0.5cm] % Minor heading such as course title
  \HRule \\[0.4cm]
  { \huge \bfseries En titt på fysiske egenskaper
  i en N-legeme kaldkollaps}\\[0.4cm]
  \HRule \\[1.5cm]
  \Large \emph{Skrevet av:}\\
  Lyder \textsc{Rumohr Blingsmo} (k.nr. 2) og Bendik \textsc{Samseth} (k.nr. 27)\\[3cm]
  {\large \today}\\[3cm]
  \vfill
\end{titlepage}

\tableofcontents
\begin{abstract}
I denne rapporten utvikler vi en N-body modell. Det vil si et system av
$N$ masser som vekselvirker kun ved gravitasjon. Spesielt studerer vi kollaps
av et slik system når alle partiklene begynner i ro. Vi sammenligner 
stabiliteten til to forskjellige løsningsmetoder, RungeKutta4 og VelocityVerlet,
ved å kikke på energibevaring i systemet. Resultatene viser blant annet
god overensstemmelse med viralteoremet, at andel partikler som skytes ut av systemet
øker med antall partikler, $N$,  og at gjennomsnittlig avstand til massesenteret minker med $N$.
Alt materiale som har blitt referert er tilgjengelig
 på Github~\cite{github-repo}. 
\end{abstract}

\begin{multicols}{2}



\section{Innledning}
I denne oppgaven skal vi se på et stjernekluster med $N$ partikler. 
I vår modell tenker
vi oss at hver partikkel representerer en eller noen få stjerner. 
Videre i rapporten kommer vi til å bruke ordene partikkel, masse, 
legeme og kropp om hverandre, men det er til stadighet egentlig
en stjerne eller samling av stjerner vi tenker oss. Først
følger en kort introduksjon av modellen vår, før vi går over til 
implementering av programmet, metoder, testing og til slutt resultater
og konklusjon.

\subsection{Krefter}
Vi
skal også anta at gravitasjon er den eneste kraften som virker:

\begin{align}
 \vec F = -\frac{GM_1M_2}{r^3}\vec r.\
\end{align}

Her er $G$ gravitasjonskonstanten, $M_i$ er massene og $r$ er avstanden mellom dem. 
Denne antagelsen er god under forutsetningen av at $r$ er relativt
stor slik at  ingen kontaktkrefter gjør seg gjeldene. Vi bruker
Newtons andre lov, og får differensialligningene 

\begin{align} \label{eq:a_i}
a_i = \sum_{j \neq i} \frac{GM_j}{{|\vec{r_i} - \vec{r_j}|}^2} , \ \ \
  i = 1,2, \cdots N 
\end{align}

der $a_i$ er akselerasjonen til masse $i$. Dette er et system av
ordinære differensialligninger, noe vi kan løse ved å anvende
numeriske metoder. I denne rapporten ser vi på RungeKutta4 og Velocity-Verlet.

\subsection{Kort om RungeKutta4~\small{\cite{RK4}}}
Metoden antar en funksjon på formen 
\begin{align*}
\dot{y} = f(t, y), \quad y(t_0) = y_0,
\end{align*}

som i vårt tilfelle er gitt ved \eqref{eq:a_i} med 

\begin{align*}
\dot y = \dot{v_i} = a_i 
\end{align*}
Vi trenger initialbetingelser for hastighet,
\begin{align*}
 v_i(t_0) = v_{i,0}.
\end{align*}
Vi velger en steglengde $h>0$ og definer 

\begin{align*}
y_{n+1} &= y_n + \tfrac{h}{6}\left(k_1 + 2k_2 + 2k_3 + k_4 \right)\\
t_{n+1} &= t_n + h \\
\end{align*}
der vi har
\begin{align*}
k_1 &= f(t_n, y_n), \\
k_2 &= f(t_n + \tfrac{h}{2}, y_n + \tfrac{h}{2} k_1), \\
k_3 &= f(t_n + \tfrac{h}{2}, y_n + \tfrac{h}{2} k_2), \\
k_4 &= f(t_n + h, y_n + hk_3).
\end{align*}

Vi gjentar så metoden en gang til for 

\begin{align*}
\dot{x_i} = v_i.
\end{align*}
I realiteten er det mulig å vektorisere prosessen slik at vi løser
både hastigheten og posisjonen til alle partiklene samtidig.

\subsection{Kort om Velocity-Verlet~\small{\cite{Velocity-Verlet}}}
\label{sec:om-VV}
Vi anvender også Velocity-Verlet på \eqref{eq:a_i}. Denne metoden fungerer for tilfeller der 
akselerasjonen ikke er en funksjon av hastigheten. 
Algoritmen for metoden er gitt som følger:

\begin{enumerate}
\item Regn ut: \begin{align*}
\vec{v}\left(t + \tfrac12\,\Delta t\right) = \vec{v}(t) + \tfrac12\,\vec{a}(t)\,\Delta t\ 
\end{align*}


\item Finn så: \begin{align*} \vec{r}(t + \Delta t) = \vec{r}(t) + \vec{v}\left(t + \tfrac12\,\Delta t\right)\, \Delta t \end{align*}


\item Regn ut: \begin{align*} \vec{a}(t + \Delta t) 
\end{align*} fra \eqref{eq:a_i} med posisjonen $ \vec{r}(t + \Delta t) $


\item Til slutt: \begin{align*} \vec{v}(t + \Delta t) = \vec{v}\left(t + \tfrac12\,\Delta t\right) + \tfrac12\,\vec{a}(t + \Delta t)\Delta t \end{align*}
\end{enumerate}

\subsection{Initialisering av systemet}
\label{sec:init-av-system}
 Vi velger å
begynne med $N$ objekter fordelt uniformt innenfor en sfære med
radius $R_0=20 ly$ (light years). Objektene gir vi en tilfeldig masse med
en Gaussisk fordeling rundt snittverdien $10M_\odot$ og et standardavvik
på $M_\odot$. Her er $M_\odot$ solmassen. Vi lar alle starte i ro; 
dette er det samme som en kaldkollaps. 

Av disse startbetingelesene er det en del som trenger noe ekstra
oppmerksomhet\footnote{Det som følger er i stor grad en oversettelse
  av et skriv laget av H. T. Ihle som ble sendt ut via epost (ikke
  listet som en referanse fordi det ikke kan vises til noe sted).}. Det er ikke helt fullstendig trivielt å lage uniformt
fordelte koordinater innenfor en sfære. Problemet er at hvis vi lar
objektene være uniformt fordelt i $r$, så vil vi ende med en mye
større tetthet rundt midten av sfæren. Og hvis vi fordeler objektene
uniformt i $\theta$ vil tettheten blir mye større rundt polene. 

Vi løser dette ved å bruke at volumelementet skal være det samme i
alle koordinatsystemer. Vi innfører koordinatene $u, v, w \in [0,1]$
og kobler disse koordinatene mot (i rekkefølge) $r,\theta$ og
$\phi$. Vi kan da skrive 
\begin{align*}
  r^2\sin\theta dr d\theta d\phi = (abc)dudvdw
\end{align*}
der $A=abc$ er en konstant. Separerer vi likningen for hvert par av
koordinater får vi likningene 
\begin{align*}
  r^2dr = adu,\ \ \sin\theta d\theta = bdv,\ \ d\phi = cdw
\end{align*}
Begynner vi med den siste og enkleste får vi 
\begin{align}
  d\phi = cdw\Rightarrow \phi = 2\pi w
\end{align}
der vi har brukt at $\phi=2\pi$ må svare til $w=1$. Samme
fremgangsmåte for $r$ og $\theta$ gir
\begin{align}
  r &= R_0 \sqrt[3]{u}\\
  \theta &= \arccos(1-2v).
\end{align}
Vi får de kartesiske koordinatene ved vanlig overgang fra sfæriske
koordinater:
\begin{align*}
  x = r\sin\theta\sin\phi\\
y = r\sin\theta\sin\phi\\
z = r\cos\theta.
\end{align*}


\subsubsection{Forventninger}
Disse initialbetingelsene blir samlet kalt sfærisk
kald-kollaps~\cite{cold-collapse}. I grensen der $N\rightarrow\infty$,
med en konstant massetetthet $\rho_0$, vil vi se at legemene kollapser
sammen i en singulæritet og danner et kontinuerlig fluid. Det kan
vises at dette vil skje etter en tid~\cite{project5-exercise}
\begin{align}
  \tau_\text{crunch} = \sqrt{\frac{ 3\pi }{ 32G\rho_0 }}\label{eq:t-crunch}.
\end{align}
Dette blir dermed en veldig naturlig tidsenhet, og vi måler alle tider
i antall $\tau_\text{crunch}$. 

Vi vil derimot ikke forvente å se en singulæritet i vår
simulering. Dette er fordi vi ser på punktmasser som kun vekselvirker
gjennom gravitasjon, i tillegg til at vi selvsagt må begrense oss til
en endelig verdi for $N$. Hva vi derimot vil forvente å se er at noen
partikler blir slynget ut av systemet (unnslipper), mens resten blir
værende i en eller annen form for bane rundt et felles massesenter. 

\section{Implementering og testing}
Hovedprogrammet er skrevet i C++, mens vi har gjort dataanalysen i Python.
Alle programmer er tilgjengelig på Github~\cite{github-repo}. 

\subsection{Klassesystem}
Vi har to klasser. En objektklasse 'Body' som inneholder posisjon, fart 
og masse for et gitt legeme. Den andre klassen er 'Universe' og initialiseres
med et gitt antall 'Body'-objekter. Denne klassen inneholder mange metoder,
som f.eks. \texttt{Universe::energy} for å regne ut energien til systemet, og 
\texttt{Universe::solve\_RK4} eller \texttt{Universe::solve\_Verlet} for å løse systemet
for en gitt tid.

\subsection{Løsningsmetoder}
Selve implementasjonen av løsningsmetodene er noe innviklet. Dette
kommer av at vi har et system av to ordinære differensialligninger,
hver med tre koordinater. Altså har vi egentlig seks diff.likninger
som skal løses samtidig. Metodene som vi har skissert tidligere endrer
ikke form, men det kan bli litt vanskelig å holde styr på alt
sammen. Derfor skal vi vi se litt nærmere på noen sider ved denne løsningsmetoden.
 Dvs., vi gjør dette for Velocity-Verlet;
RungeKutta4 er løst på en analog måte. Velocity-Verlet er implementert i
\texttt{Universe} slik som vist i vedlegg \ref{lst:solver-RK4}. 

Vi begynner med å definere flere $N\times 7$ matriser. Dette svarer til
en rad per legeme, der kolonnene er de tilhørende verdiene av (i rekkefølge) $x,y,z,v_x,v_y,v_z$
og $M$. For eksempel matrisen $y_i$ er på formen
\begin{align*}
  y_i = \begin{pmatrix}
    x_1&y_1&z_1&v_{x_1}&v_{y_1} &v_{z_1} & m_1\\
    x_2&y_2&z_2&v_{x_2}&v_{y_2} &v_{z_2} & m_2\\
    \vdots&&\ddots&&&&\vdots\\
    x_N&y_N&z_N&v_{x_N}&v_{y_N} &v_{z_N} & m_N\\
  \end{pmatrix}
\end{align*}

Til å begynne med fyller vi $y_i$ med data fra den initielle tilstanden til systemet.
Vi går så inn i tidsløkken. Her brukes en funksjon
\texttt{derivative(y\_i, a\_i, n\_bodies)} som gjør at matrisen $a_i$ fylles med de
deriverte verdiene av matrise $y_i$. Det vil si at posisjonen i $y_i$
får tilsvarende element i $a_i$ satt som hastigheten, og hastigheten i
$y_i$ blir tilsvarende satt til akselerasjonen i $a_i$. 

\begin{align*}
  a_i = \begin{pmatrix}
    v_{x_1}&v_{y_1} &v_{z_1}&a_{x_1}&a_{y_1} &a_{z_1} & m_1\\
    v_{x_2}&v_{y_2} &v_{z_2}&a_{x_2}&a_{y_2} &a_{z_2} & m_2\\
    \vdots&&\ddots&&&&\vdots\\
    v_{x_N}&v_{y_N} &v_{z_N}&a_{x_N}&a_{y_N} &a_{z_N} & m_N\\
  \end{pmatrix}
\end{align*}

Med disse elementene følger vi så algoritmen for Velocity-Verlet som
er skissert i seksjon \ref{sec:om-VV}. Samme type tanke blir brukt for
å implementere Rungekutta4.


\subsection{Test av 2body-system}
For å forsikre oss om at algoritmen fungerer begynner vi med å simulere
et system med kun to legemer. I et slikt enkelt system er det lett å 
sammenligne resultatene med ens egen intuisjon av systemet. Det finnes
også en analytisk løsning. I figur \ref{fig:2body} kan vi se resultatet 
av en kjøring. Vi ser at banene er ellipsoider, slik vi forventer for
to-legeme-systemet. Dette gir oss selvtillit til å jobbe videre
med et større system.

\end{multicols}

\begin{figure}[!h]
  \centering
  \includegraphics[scale=0.4]{../fig/2body.png}
  \caption{\label{fig:2body} En figur som viser to legemer sin bane, når eneste
    kraft som virker på dem er gravitasjon. Partikkel 0 starter i 
    punktet $(0,0,0)$ og partikkel 1 i $(5,0,0)$. De har hastigheter 
    henholdsvis $\vec{v} = (0, -0.1, 0) $ og $\vec{v} = (0, 0.1, 0)$  }
\end{figure}


\begin{multicols}{2}

\subsection{Første test av $N$-body}
\label{sec:forste-test-epsilon}
Et N-body-system er dessverre ikke like lett å se for seg som
2-body-systemet. Derfor er vi nødt til å finne på noe lurere enn
bare et stillbilde av partikkelbanene. Vi begynner med å animere systemet 
\footnote{Tilgjengelig på \cite{github-repo} under navnet
animate\_pos.py}, og ser fort at det skjer noe rart når massene kommer veldig
nærme hverandre. Noen legemer blir 'skutt' ut i stor fart. Dette kommer fra 
likning \eqref{eq:a_i}. Akselerasjonen, $a$, blir veldig
stor når $|\vec{r_i} - \vec{r_j}| \rightarrow 0$.
Som en korreksjon til dette legger vi til et feilledd i kraften

\begin{align}
F_{mod} = -\frac{GM_1M_2}{r^2 + \epsilon},
\end{align}

der $\epsilon$ gjør at $F_{mod}$ ikke går i lufta når $r^2 \rightarrow 0$. Vi 
rettferdiggjør denne korreksjonen ved at masser i den virklige
verden aldri vil komme så nærme hverandre fordi de ikke er punktmasser.
Vi setter $\epsilon = 10^{-3}$, som vi senere vil se gir
gode resultater. Denne verdien er funnet etter prøving og feiling. 
$\epsilon > 10^{-3}$ gir oss en uønsket stor dempning i kraften, og 
$\epsilon < 10^{-3}$ gir fort samme problem som vi begynte med, deling på
veldig små tall. Vi merker oss at $\epsilon = 10^{-3} {ly}^2$ (light years) er et veldig 
stort tall, til sammenligning er $R_{solar}^2 \sim 10^{-15} {ly}^2$. Dette høres kanskje katastrofalt
ut, men vi ser at det er en nødvendighet for å få god presisjon i beregningene våre.

\subsection{Energibevaring}
Selv om animasjoner alltid er flott, tenker vi det kan være lurt å ha en mer
'matematisk' test på systemet vårt. Dette gjør oss også mer egnet til
å gjøre en vurdering av hvilken av de løsningsmetodene som er best.
I et slikt ideelt system med bare 
gravitasjon, en konservativ kraft, er det naturlig å se på energibevaring. Vi får
energien

\begin{align}
E_{p} = - \sum_{i = 1}^{N} \sum_{j = i+1}^{N} \frac{GM_jM_i}{{||\vec{r_i} - \vec{r_j}||}} \\
E_{k} = \sum_{i} \frac{1}{2}M_i|\vec{v_i}|^2 \\
E_{tot} = E_{p} + E_{k} = \text{konstant}
\end{align}

I vedlegg \ref{lst:methodcomptable} ser vi en sammenligning av energibevaringen til 
våre to metoder. Tabellen viser den relative feilen i energibevaring,
samt tiden som ble brukt, for begge metoder og for flere tider og
steglengder.  Vi ser begge metodene oppnår god presisjon med tilstrekkelig
liten steglengde. RK4 bruker jevnt over omtrent dobbelt så lang tid
som Verlet. For å spare kjøretid velger vi derfor å bruke 
Velocity-Verlet. Det kan hende vi mister noe presisjon sammenliknet med
å bruke RK4, men vi ønsker å kunne kjøre beregninger for lange tider
og store systemer. Det betyr at så lenge metoden er stabil med tiden,
så er kjøretid mer viktig enn kortsiktig presisjon. Tabellen i vedlegg \ref{lst:methodcomptable}
 viser oss også at vi trenger ganske små tidssteg for å få god
energibevaring.\footnote{Merk at tabellen i vedlegg
  \ref{lst:methodcomptable} er laget med data generert med en
  tidligere versjon av hovedprogrammet. Vi mener at de endringene som
  har blitt gjort siden den gang ikke påvirker resultatene, men det
  kan hende vi nå ville oppnådd litt forskjellige verdier. Vi fikk
  tekniske problemer med programmet som laget tabellen og har ikke
  fått laget en ny utgave (mystisk ``Floating point exception''
  krasjer programmet). Vi mener i alle tilfeller at konklusjonen i
  forhold til metode er gyldig.}

\subsection{Utskytning av partikler}
Rundt tidspunktet $\tau_{crunch}$ og i en stund etter vil partikler skytes ut
av det opprinnelige kuleskallet. Dette er partikler som ikke vil
være en del av vår likevektstilstand, og vi ønsker derfor å ha muligheten
til å se bort fra dem. Men hvordan kan vi identifisere en partikkel
som er skutt ut? Vi vet at en partikkel med $E_{tot} > 0$ har $E_K > |E_p|$, og
vil derfor unnslippe galaksehopen. Å ha dette kriteriet for utskutte partikler
viser seg derimot å være ustabilt. Se
f.eks. figur \ref{fig:energivariasjon} der vi kan se at den gjennomsnittlige
energien varierer lite, selv om den 'lokale' energien noen ganger gjør et hopp.
 På grunn av denne lokale numeriske ustabiliteten ser vi oss nødt til å gjøre
noe mer robust enn kun å se på energien. Vi observerer at en utskutt 
partikkel alltid vil forsvinne mot 'det fjerne', og derfor at 
$r_{utskutt} > R_0$. Vi bestemmer oss derfor for at i vårt program har
en utskutt partikkel ikke \emph{bare} positiv energi, $E > 0$, men befinner seg også
utenfor kuleskallet vi begynte med. 

I programmet vårt fjerner vi de utskutte
partiklene ved å sette massen og hastighetene til null, og å 'gjemme' dem bort i et hjørne. 


\subsection{Definisjon av likevekt}
En konsekvens av at vi fjerner partikler er at energien ikke lenger er bevart
gjennom hele simuleringen. Til gjengjeld får vi en veldig fin måte å definere
likevekt på. Det følger naturlig fra forrige avsnitt at likevektstilstanden
er når ingen flere partikler fjernes, det vil si energien ikke forandrer seg
betydelig lenger.

\end{multicols}
\begin{figure}[!ht]
  \centering
  \includegraphics[width=\textwidth]{../fig/energivariasjon.png}
  \caption{\label{fig:energivariasjon} En figure som viser  }
\end{figure}

% \includemedia[
%   width=0.4\linewidth,
%   height=0.3\linewidth,
%   activate=pageopen,
%   addresource=../fig/animation.mp4,
%   flashvars={source=../fig/animation.mp4}
% ]{}{VPlayer.swf}

\cleardoublepage
\begin{multicols}{2}

\section{Resultater}
Vi har som nevnt i seksjon \ref{sec:init-av-system} simulert et system
av $N$ partikler uniformt fordelt i en sfære med radius $R0$. Vi har
laget noe animasjoner som viser tidsforløpet til
systemet~\cite[\texttt{project5/fig/}]{github-repo}. Fra animasjonene
ser vi følgende:
\begin{enumerate}
  \item Alle partiklene begynner i ro, og beveger seg sakte inn mot
    origo.
  \item Alle partiklene blir most sammen i sentrum av systemet.
  \item Partiklene spretter tilbake igjen, noen helt ut av systemet.
  \item Massenteret beveger seg vekk fra origo med tilnærmet konstant
    hastighet slik vi ser i figur \ref{fig:massesenter}.
  \item Vi står til slutt igjen med et redusert antall partikler som
    ser ut til å bevege seg noen lunde stabilt rundt i massesenteret.
\end{enumerate}


\end{multicols}
\begin{figure}[ht!]
  \centering
  \includegraphics[width=\textwidth]{../fig/massesenter.png}
  \caption{\label{fig:massesenter} En figur som viser massesenterets
  avstand fra origo som funksjon av tid etter likevekt er nådd. Vi ser 
  at massesenteret beveger seg vekk med tilnærmet konstant fart.}
\end{figure}
\begin{multicols}{2}



\subsection{Likevekt}
Fra animasjonen ser vi at en stund etter $\tau_\text{crunch}$
stabiliserer systemet seg i en slags likevekt. For å se litt mer kvantitativt på
dette kan vi se på energien til systemet som en funksjon av tid. Figur
\ref{fig:energi-vs-tid-med-utskytning} viser dette. 

\end{multicols}
\begin{figure}[ht!]
  \centering
  \includegraphics[width=\textwidth]{../fig/energy_plot_with_ejection.png}
  \caption{\label{fig:energi-vs-tid-med-utskytning} Systemets
    totalenergi som funksjon av tid med masseutskytelse. Vi legger
    særlig merke til at en liten tid etter $\tau_\text{crunch}$ får vi
  en voldsom reduksjon i energien til systemet. Vi ser så at etter en
  tid ser energien ut til å stabilisere seg på et lavere nivå.}
\end{figure}
\begin{multicols}{2}
Vi ser fra figuren at energien er konstant i en tid litt større enn
$\tau_\text{crunch}$. Deretter ser vi en veldig bratt, og stykkvis
nedgang i energien. Dette er fordi alle partiklene har blitt most
sammen ved tiden $\tau_\text{crunch}$, hvor noen har fått positiv
totalenergi og beveger seg ut fra systemet. Grunnen til at nedgangen
ser ut til å komme forsinket på figuren er på grunn av at vi
krever at partikkelen også er utenfor radiusen $R_0$, noe som selvsagt
tar en viss tid. Deretter bruker systemet litt tid på å stabilisere
og etter omtrent $t = 2.5\,\tau_\text{crunch}$ ligger energien igjen
flatt. Dette er likevekt. 

Grunnen til at totalenergien blir mer negativ når vi tar ut partiklene
følger logisk av at vi bare tar ut partikler som har netto positiv
energi. Tar vi vekk disse blir totalen åpenbart mer negativ. 


\subsubsection{$N$-avhengighet}
Vi kan lure på hvordan dette endrer seg med $N$. I figur
\ref{fig:energy-many-N} ser vi tidsforløpet til systemenergien for
forskjellige antall partikler. Først kan vi merke oss at alle
systemene begynner å miste partikler etter omtrent samme tid (en liten
tid etter $\tau_\text{crunch}$), og at alle ser ut til å bruke samme
tid på å nå et nytt stabilt energinivå. 

\end{multicols}
\begin{figure}[!ht]
  \centering
  \includegraphics[width=\textwidth]{../fig/energy_plot_with_ejection_many_N.png}
  \caption{\label{fig:energy-many-N} Tidsforløpet til systemenergien
    for flere verdier av $N$. Vertikal akse er normalisert mot
    initialenergien for hver $N$. Vi ser at likevekten etableres mer eller
  mindre like fort uavhengig av hvor mange partikler som er med.}
\end{figure}
\begin{multicols}{2}

I figur \ref{fig:prosentandel-utskutte} kan vi se hvordan prosentandelen
av partiklene som blir skutt ut endrer seg som funksjon av N. Vi ser
store variasjoner i punktene, men en klar trend som viser at en større
prosentandel av partiklene vi begynner med blir skutt ut for større N. Som en
konsekvens av dette følger det også at en større prosentandel av energien vil forsvinne
for større $N$.

\end{multicols}
\begin{figure}[ht!]
  \centering
  \includegraphics[width=\textwidth]{../fig/prosentandel.png}
  \caption{\label{fig:prosentandel-utskutte} En figur som viser andel av partikler
  som blir skutt ut som funksjon av $N$. Vi ser store variasjoner i punktene, men en klar
  trend at større $N$ gir større andel utskutte partikler. Slik ser vi også at en større
  andel av energien vil forsvinne når $N$ øker.}
\end{figure}
\begin{multicols}{2}



\subsection{Viralteoremet}
En interessant ting å se på er distribusjonen av kinetisk- og
potensiell energi. Viralteoremet sier noe om akkurat dette~\cite{viralteoremet-wiki}:

\begin{viralthm}
  For et gravitasjonelt bundet system i likevekt har vi 
\begin{align}
  2\mean{K} = -\mean{V},
\end{align}
der $\mean{K}$ og $\mean{V}$ er tidsmidled av henholdsvis
kinetisk- og potensiell energi.
\end{viralthm} 

\end{multicols}
\begin{figure}[!ht]
  \centering
  \includegraphics[width=\textwidth,height=300px]{../fig/viral.png}
  \caption{\label{fig:viral} Forholdet mellom potensiell- og kinetisk
 energi etter at vi har nådd likevekt. Figuren viser både tidsforløpet
til fordelingen og tidsmidlet. Vi ser at vi får en verdi som ligger
svært nære den verdien vi forventer fra Viralteoremet.}
\end{figure}
\begin{multicols*}{2}

Figur \ref{fig:viral} viser en graf av forholdet mellom potensiell- og
kinetisk energi etter at vi har nådd likevekt.\footnote{Plottet er
  laget med et program kalt
  \texttt{viralTheorem.py}~\cite{github-repo}} Det ser ut som om
forholdet i simuleringen oscillerer rundt den ideelle verdien fra
Viralteoremet. I høyre hjørne i figuren er tidsmidlet av forholdet
regnet ut, og som vi ser får vi en verdi svært nære 2. 

Vi har beregnet resultatene i figur \ref{fig:viral} ved å ta midlet av
alle partiklenes energi på et gitt tidspunkt. Dette kan vi gjøre ved
den ergodiske hypotesen. Tidsmidlet av disse
verdiene (tidsmidlet av partikkelmidlet) er det som er vist i høyre
hjørne på figuren.

Dette er gode nyheter, og tyder på at systemet vårt følger
naturens lover på tross av de forenklingene som har blir
gjort. F.eks. styrker dette begrunnelsen for vårt valg av $\epsilon$
(seksjon \ref{sec:forste-test-epsilon}) ikke har gitt urealistiske
resultater. 

\subsection{Partikkeldistribusjon}
En svært interessant ting å se på er hvordan partiklene er fordelt. I
denne sammenheng skal vi se på partikkeldistribusjonen $n(r)$ (antall legemer ved
radius $r$) og partikkeltettheten $\rho(r)$ (antall legemer innenfor
et volum $V=\frac{4}{3}\pi r^3$). 

Figur \ref{fig:nummertetthet} viser partikkeldistribusjonen som
funksjon av avstanden til massesenteret. Vi ser at de aller fleste
partiklene ligger nære massesenteret, og at fordelingen faller raskt
når $r$ øker. Dette betyr at partiklene i stor grad er hopet tett
sammen rundt massesenteret, mens vi har noen få partikler som befinner
seg lenger ut. 

Figur \ref{fig:partikkeltetthet} viser partikkeltettheten, igjen som
en funksjon av avstanden til massesenteret. Denne er regnet ut som
$N(R)/\frac{4}{3}\pi r^3$, der $N(r)$ er alle partiklene innefor en
sfære med radius $r$. Her har vi normalisert x-aksen med $N^{-1/3}$ og
y-aksen med $N^{2}$. Dette gjør vi for å få frem likheten med følgende
modell:
\begin{align}
  \Xi(r) = \frac{n_0}{1+\left(\frac{ r }{ r_0 }\right)^4}
\end{align}
 Modellen og
skaleringen er tatt fra artikkelen i referanse~\cite{cold-collapse}. Selv om vi ikke har fått
datapunkter langs hele modellkurven slik det står i artikkelen, så er
vi overrasket over hvor tilsynelatende godt vi har fått modellkurven
til å passe med datasettet.

\subsubsection{Gjennomsnittlig avstand til massesenter}
I figur \ref{fig:gjennomsnittlig} ser vi at den gjennomsnittlige avstanden til 
massesenteret minker som funksjon av $N$. Dette kan vi forklare med at kjernen
blir mer massiv ved større $N$, og derfor også trekker partiklene tetter på seg.
Det er stor spredning i dataene, men det kan godt være at trenden ikke er fullt
så tydelig for små $N$, slik vi har her. Til sammenligning kan vi se på 
~\cite[figur 1]{cold-collapse} der et lignende plot er laget, men med et mye større
system. 


\end{multicols}. 

\begin{figure}[ht]
  \centering
  \includegraphics[width=\textwidth]{../fig/gjennomsnittligavstand.png}
  \caption{\label{fig:gjennomsnittlig} Plot av gjennomsnittlig avstand
    til massesenter etter likevekt, som funksjon av $N$. Vi ser en nedadgående
  trend. Det kan vi forklare ved at kjernen blir mer massiv og trekker sterkere
  på legemene når $N$ blir større.}
\end{figure}


\begin{multicols}{2}




\end{multicols}. 

\begin{figure}[ht]
  \centering
  \includegraphics[width=\textwidth]{../fig/nummertetthet.png}
  \caption{\label{fig:nummertetthet} Histogram som viser fordelingen
    av partikler som funksjon av avstanden til massesenteret. Vi ser
    at de aller fleste partiklene ligger nære massesenteret, og
    fordelingen faller raskt av for større $r$.}
\end{figure}

\begin{figure}[ht]
  \centering
  \includegraphics[width=\textwidth]{../fig/partikkeltetthet.png}
  \caption{\label{fig:partikkeltetthet} Graf som viser
    partikkeltettheten $\rho(r)$ for mange forskjellige verdier av
    $N$. Over er det tegnet inn modellkurven som ser ut til å passe
    med datasettet (se tekst for forklaring).}
\end{figure}

\begin{multicols}{2}

\section{Konklusjon}
Vi har fått en grundig innføring i N-body-systemet, fysikken bak denne
og også hvordan den oppfører seg. Vi har fått muligheten til å
bruke Velocity-Verlet og RungeKutta4 på større systemer, og dermed
også en mye bedre forståelse av disse to algoritmene.

Prosjektet har vært lærerikt ved at vi har blitt kjent med
klasseprogrammering i C++ og også mer komfortable med databehandling
i Python. Mange-legeme-systemet har også vist oss at
tilsynelatende 'kaotiske' resultater, ofte kan tilnærmes
med en overraskende enkel funksjon. I dette tilfellet eksempelvis figur
\ref{fig:viral} og \ref{fig:nummertetthet} som vi har tilpasset en 
forventet kurve med gode resultater.

Nok en gang har vi fått oppleve viktigheten av grundig testing
av programmet, kanskje spesielt i start-fasen av et nytt prosjekt. Da 
kaster man ikke bort tiden ved å jobbe videre på noe som er feil. Senest
dagen før frist oppdaget vi nye bugs i programmet. Som vanlig er vi
selvfølgelig begrenset av både tid og maskinkraft, og skulle ønske vi 
levde i en perfekt verden der vi kunne simulert for et stort antall partikler,
i lange tider, med små tidssteg. Slik er det dessverre ikke, 
men allikevel har vi fått gode nok resultater til å se fysikken bak mange-legeme-problemet.







\end{multicols*}
\clearpage
\printbibliography
\clearpage
\section{Vedlegg}
\begin{itemize}
  \item[]\lstinputlisting[language=c++, caption={C++ metode som bruker
      Velocity Verlet til å tidsutvikle systemet.},
    label={lst:solver-RK4}]{solversnippet.txt}
\pagebreak
  \item[]\lstinputlisting[basicstyle=\small,caption={En tabell som sammenligner relativ energiendring for de to metodene
RungeKutta4 og VelocityVerlet. Energiendringen regnes ut fra systemet starter
i ro ved $t_0 = 0$ til $T$ for to forskjellige steglengder
$h$.  Kjøretiden, $T$, måles i enhet $\tau_{crunch}$. Ideelt sett vil vi ha $\Delta E = 0$. Vi ser Rungekutta bruker omtrent dobbelt så lang tid for en gitt h,
uten å gi noe særlig bedre presisjon.},
    label={lst:methodcomptable}]{../src/data/methodCompTable.txt}
\end{itemize}

\end{document}
%%% Local Variables:
%%% mode: latex
%%% TeX-master: t
%%% End:
