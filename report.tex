\documentclass[11pt]{article}
\usepackage{preamble}
\usepackage{fancyhdr} % Custom headers and footers
\pagestyle{fancyplain} % Makes all pages in the document conform to the custom headers and footers
\fancyhead[C]{\footnotesize \textit{FYs3150 Prosjekt
    2 $ \, $}} % No page header - if you want one, create it in the same way as the footers below 
\fancyhead[R]{}
\fancyhead[L]{}
\fancyfoot[L]{} % Empty left footer
\fancyfoot[C]{} % Empty center footer
\fancyfoot[R]{\thepage} % Page numbering for right footer
\renewcommand{\headrulewidth}{0pt} % Remove header underlines
\renewcommand{\footrulewidth}{0pt} % Remove footer underlines
\setlength{\headheight}{21pt} % Customize the height of the header
\begin{document}
  % make title page
\begin{titlepage}
  \newcommand{\HRule}{\rule{\linewidth}{0.5mm}}
  \center
  \textsc{\LARGE Universitetet i Oslo}\\[1.5cm] % Name of your university/college
  \textsc{\Large }\\[0.5cm] % Major heading such as course name
  \textsc{\large FYS3150}\\[0.5cm] % Minor heading such as course title
  \HRule \\[0.4cm]
  { \huge \bfseries Prosjekt 2 }\\[0.4cm] % Title of your document
  \HRule \\[1.5cm]
  \Large \emph{Skrevet av:}\\
  Lyder \textsc{Rumohr Blingsmo} og Bendik \textsc{Samseth}\\[3cm]
  {\large \today}\\[3cm]
  \vfill
\end{titlepage}


\section*{Schödingers likning for to elektroner i en tredimensjonal
  harmonisk oscillator brønn}
\begin{abstract}
  Denne oppgaven skal løse Schödingers likning for to elektroner i en tredimensjonal
  harmonisk oscillator brønn med og uten en
  Coulomb-vekselvirkning. For å gjøre dette blir likningen skrevet om
  til diskret form som en eigenverdilikning. Denne blir så løst ved
  hjelp av en implementasjon av Jacobis metode. 
\end{abstract}


Vi antar at to elektroner befinner seg i et tredimensjonalt harmonisk
oscillator (HO) brønnpotensial, og at de vekselvirker gjennom
Coulombpotensialet mellom dem. Vi antar sfærisk symmetri. 

Vi er først interessert i løsningen på den radielle delen av
Schödingers likning (SL) for et elektron. Denne likningen kan skrives
som 
\begin{align}
  -\frac{ \hbar^2 }{ 2m }\left( \frac{ 1 }{ r^2 }\frac{ d }{ dr }r^2
  \frac{ d }{ dr } - \frac{ l(l+1) }{ r^2 }   \right) R(r) + V(r)R(r)
  = ER(r).\label{eq:radiell-SL}
\end{align}


\end{document}


%%% Local Variables:
%%% mode: latex
%%% TeX-master: t
%%% End:
