\documentclass[11pt]{article}
\usepackage[norsk]{babel}
\usepackage[utf8]{inputenc}
\usepackage{alltt}
\usepackage{listings}
\usepackage{hyperref}
\usepackage{color}
\usepackage{amsmath}
\usepackage{amssymb}
\usepackage{amsthm}
\usepackage{wasysym}
\usepackage{lipsum}
\usepackage{sectsty}
\usepackage{fullpage}
\usepackage{graphicx}
\usepackage{import}
\usepackage{cancel}
\usepackage{pdfpages}
\usepackage{siunitx}
        \DeclareSIUnit\gauss{G}
	\sisetup{exponent-product = \cdot}	% Prikk som multiplikasjonstegn (i stedet for kryss).
 	\sisetup{output-decimal-marker  =  {.}}	% Punktum som desimalskilletegn (i stedet for komma).
 	\sisetup{separate-uncertainty = true} 	% Pluss-minus-form på
                                % usikkerhet (i stedet for parentes). 
\usepackage{pgfplots}
\pgfplotsset{compat=1.9}
\usepackage{comment}

% Denne setter navnet på abstract til Sammendrag
% \renewenvironment{abstract}{\global\setbox\absbox=\vbox\bgroup
% \hsize=\textwidth\def\baselinestretch{1}%
% \noindent\unskip\textbf{Sammendrag}
% \par\medskip\noindent\unskip\ignorespaces}
% {\egroup}

% Disse kommandoene definerer hvor stor andel av siden som kan være dekket av figurer. Kan gjøre det enklere å plassere figurer.
\setcounter{totalnumber}{15}
\renewcommand{\textfraction}{0.05}
\renewcommand{\topfraction}{0.95}
\renewcommand{\bottomfraction}{0.95}
\renewcommand{\floatpagefraction}{0.35}


%\renewcommand{\theequation}{(\thesubsection.hei\arabic{equation})}
\newcommand\numberthis{\addtocounter{equation}{1}\tag{\theequation}}
\renewcommand{\thesection}{\arabic{section}}
\renewcommand{\thesubsection}{\alph{subsection})}
%\renewcommand{\thesubsection}{\thesection.\alph{subsection})} % subsection is now 1.a), 1.b)... 2.a)...

%\allsectionsfont{\centering \normalfont} % Make all sections centered, the default font

\usepackage{fancyhdr} % Custom headers and footers
\pagestyle{fancyplain} % Makes all pages in the document conform to the custom headers and footers
\fancyhead[C]{\footnotesize \textit{FYS3150 Prosjekt 1\\ $ \,$}} % No page header - if you want one, create it in the same way as the footers below
\fancyhead[R]{} \fancyhead[L]{} \fancyfoot[L]{} % Empty left footer
\fancyfoot[C]{} % Empty center footer
\fancyfoot[R]{\thepage} % Page numbering for right footer
\renewcommand{\headrulewidth}{0pt} % Remove header underlines
\renewcommand{\footrulewidth}{0pt} % Remove footer underlines
\setlength{\headheight}{21pt} % Customize the height of the header

% \numberwithin{equation}{subsection} % Number equations within sections (i.e. 1.1, 1.2, 2.1, 2.2 instead of 1, 2, 3, 4)
\numberwithin{figure}{section} % Number figures within sections (i.e. 1.1, 1.2, 2.1, 2.2 instead of 1, 2, 3, 4)
\numberwithin{table}{section} % Number tables within sections (i.e. 1.1, 1.2, 2.1, 2.2 instead of 1, 2, 3, 4)


\definecolor{yellow}{rgb}{0.7,0.8,0}
%\definecolor{blue}{rgb}{0,0,0.8}
\definecolor{blue}{rgb}{0.3,0.3,0.9}
\definecolor{green}{rgb}{0,0.3,0}
\lstloadlanguages{Python} 
\lstloadlanguages{MATLAB}
\lstset{%frame=single, % Single frame around code
		otherkeywords={self},
        keywordstyle=\color{blue},
        commentstyle=\color{green},
        stringstyle=\color{green}, % Strings are green
        showstringspaces=false, % Don't put marks in string spaces
        tabsize=5, % 5 spaces per tab        
        numbers=none, % Line numbers on left
        %firstnumber=1, % Line numbers start with line 1
        stepnumber=1 % Line numbers go in steps of 5
}


\newcommand{\horline}{
\begin{center}
\line(1,0){350}
\end{center}
}


% common commands
\newcommand{\pd}[2] {\frac{\partial #1}{\partial #2}}
\renewcommand{\div}[1] {\nabla\cdot\vec{#1}}
\newcommand{\curl}[1] {\nabla\times\vec{#1}}
\renewcommand{\vec}{\mathbf} % bold face for vectors
\newcommand{\e}[1] {\times 10^{#1}}
\newcommand{\mean}[1] {\langle #1 \rangle}


\begin{document}
% make title page
\begin{titlepage}
\newcommand{\HRule}{\rule{\linewidth}{0.5mm}}
\center
\textsc{\LARGE Universitetet i Oslo}\\[1.5cm] % Name of your university/college
\textsc{\Large }\\[0.5cm] % Major heading such as course name
\textsc{\large FYS3150}\\[0.5cm] % Minor heading such as course title
\HRule \\[0.4cm]
{ \huge \bfseries Prosjekt 1 }\\[0.4cm] % Title of your document
\HRule \\[1.5cm]
\Large \emph{Skrevet av:}\\
Lyder \textsc{Rumohr Blingsmo} og Bendik \textsc{Samseth}\\[3cm]
{\large \today}\\[3cm]
\vfill
\end{titlepage}


\begin{abstract}
I dette prosjektet skal vi kjent med ulike vektor- og
matriseoperasjoner. Vi skal benytte C++ for størsteparten av
beregningene i et forsøk på å bli bedre kjent med språket. Vi ser på
andreordens lineære differensialligninger, spesielt ser vi på den
generelle endimensjonelle Poisson ligningen med Dirichlet
randbetingelser. Vi ser på flere måter å løse slike systemer, og
analyserer forskjellene med tanke på kjøretid og nøyaktighet.
\end{abstract}


Vi har gitt den generelle formen til den endimensjonelle
Possionsligningen:
\begin{align}
  -u''(x) = f(x)\label{eq:poisson}
\end{align}
med $x \in (0,1)$ og $u(0) = u(1) = 0$.

\subsection{}

For å komme frem til en numerisk løsning til \eqref{eq:poisson}
definerer vi den diskretiserte tilnærmingen til $u(x)$ som $v_i$ med
x-verdier $x_i = ih$ på intervallet $x_0 = 0$ og $x_{n+1} =
1$. Avstanden mellom hver $x_i$ defineres som $h =
1/(n+1)$. Randbetingelsene blir da $v_0 = v_{n+1} = 0$.

Ligning \eqref{eq:poisson} viser at vi trenger den dobbeltderiverte av
$u(x)$. For $v_i$ kan vi tilnærme dette som 

\begin{align}
  - \frac{ v_{i+1} + v_{i-1} - 2v_i }{ h^2 } =
  f_i\hspace{1cm}\text{for } i = 1,\dots,n.
\end{align}
der $f_i = f(x_i)$. Vi setter 
\begin{align*}
      {\bf A} = \left(\begin{array}{cccccc}
                           2& -1& 0 &\dots   & \dots &0 \\
                           -1 & 2 & -1 &0 &\dots &\dots \\
                           0&-1 &2 & -1 & 0 & \dots \\
                           & \dots   & \dots &\dots   &\dots & \dots \\
                           0&\dots   &  &-1 &2& -1 \\
                           0&\dots    &  & 0  &-1 & 2 \\
                      \end{array} \right)
\end{align*}
Da ser vi at $\vec A \vec v$ blir lik 
\begin{align*}
  \vec A \vec v = -v_{i+1} - v_{i-1} + 2v_i\hspace{1cm}\text{for } i = 1,\dots,n
\end{align*}
når vi har at $v_0 = v_{n+1} = 0$ fra Dirichlet-randbetingelsene. Hvis
vi definerer $\vec d$ ved $d_i = h^2f_i$ kan vi skrive
\eqref{eq:poisson} som 
\begin{align}
  \vec A \vec v = \vec d.
\end{align}
 

Vi skal videre anta at $f(x) = 100 e^{-10x}$. Da har
\eqref{eq:poisson} en eksakt løsning gitt som $u(x) = 1 -
\left(1-e^{-10}\right) - e^{-10x}$. Vi skal bruke denne til å
sammenlikne med vår numeriske lønning i senere oppgaver.

\subsection{}








\end{document}
%%% Local Variables: 
%%% mode: latex
%%% TeX-master: t
%%% End: 


















